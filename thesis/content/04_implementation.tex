\chapter{Implementation}
\label{cha:implementation}

In this chapter the construction of a prototype will be described. 
This section will give an overview about the actual implementation details. It shall enable the user to use and configure the prototype. First, the overall structure of the codebase will be explained. Then, more details about the actual implementation of the smart contracts are described. The marketplace interface is explained at last. 

\section{codebase}

\section{Structure of Smart Contracts}

An overview of the smart contracts is shown in Figure 4. Various functionalities of the prototype has been split into different contracts and  inheritance was being used as a means to combine them meaningfully. 
 
Because Identity Management is becoming more and more of an issue, my prototype wants to enable self souvereign identity. 
This means, that users should have full control over their data and their privacy settings. A public private key pair already represents an identity. To conform to the W3C standard, that key has to be referenced with certain data. 

ERC1056 has been chosen, because it offers decentralized management of identifiers. 

